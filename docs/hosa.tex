\documentclass{article}  
\input{header}

\htmltitle{Songlib: Hosa Tracklink recording}

\title{Songlib: tracklink\\
\date{Revision Date: \today}}

\author{written by: Song Li Buser}

\begin{document}

\maketitle

\W\subsubsection*{\xlink{Printable Version}{hosa.pdf}}
\W\htmlrule

\section*{Introduction}

This tutorial assumes Ubuntu Linux running the ALSA Sound Server,
although this should work for other versions of Linux and
other sound servers, with appropriate modifications.

You should have a shell script named {\it record}
that uses the alsa audio driver.
You will also need to install {\it ecasound} and {\it sox}:

\begin{verbatim}
    sudo apt-get install ecasound sox
\end{verbatim}

\begin{description}

\item[Step 1]

Using the Hosa Tracklink USB cable, connect your microphone
to your laptop.  If you run the following command:

\begin{verbatim}
    record
\end{verbatim}

without any arguments, you should see a message similar to:

\begin{verbatim}
    you must specify a sound card number!
     0 [PCH            ]: HDA-Intel - HDA Intel PCH
                          HDA Intel PCH at 0xe3d40000 irq 44
     1 [Interface      ]: USB-Audio - Tracklink USB Interface
                          Hosa Technology, Inc. Tracklink USB Interface at ...
        \end{verbatim}

Card 0 is your internal sound card while card 1
should be the Hosa Tracklink.

\item[Step 2]

Turn the volume button on the Hosa Tracklink all the way up (clockwise).
This control is on the top of the small brick in the middle of the 
cable.
The Tracklink supports both condensor and dynamic mics.
For condensor mics, turn on the phantom power; the phantom power
switch is on the side of the brick. If the phantom power is on,
the 48V LED will light up.

\item[Step 3]

Now you need to set the levels. In one terminal window, run the command:

\begin{verbatim}
    alsamixer -c 1
\end{verbatim}

In another window, run the command:

\begin{verbatim}
    monitor 1
\end{verbatim}

Plug in some good headphones into your computer and then
speak into the mic. You should hear your voice through the headphones,
with a slight delay.
Using {\it alsamixer},
make sure your master, pcm, and headphones are near 100\%.
Now you want to adjust the volume control on the Tracklink
until you have the loudest result in your headphones WITHOUT any
annoying buzz or hiss.
Note that only one side of your head phones
will change during this volume adjustment since the mic is monophonic.
Stop monitoring in the monitoring window with a \verb!<Ctl>-c!.

\item[Step 4]

To record, enter the command:

\begin{verbatim}
    record 1
\end{verbatim}

At the {\it ecasound} prompt, press \verb!t! to start recording, \verb!s! to
stop recording, and \verb!q! to quit.
Note that if you want to record with the built-in
device, you would enter the command:

\begin{verbatim}
    record 0
\end{verbatim}

After recording,
play the file {\it track.wav} back through the laptop speakers
to make sure you have recorded properly:

\begin{verbatim}
    play track.wav
\end{verbatim}

If you experience clipping, reduce the loudness of your voice or
reduce the volume control on the Tracklink.
If you want to save the recording, save {\it track.wav}
under another name, since subsequent recordings will
overwrite {\it track.wav}.

\end{description}

\end{document}
