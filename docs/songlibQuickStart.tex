\documentclass{article}  
\input{header}

\htmltitle{songlib: quickstart}

\title{Songlib: quickstart\\
\date{Revision Date: \today}}

\author{Song Li Buser}

\begin{document}

\maketitle

\W\subsubsection*{\xlink{Printable Version}{songlibQuickStart.pdf}}
\W\htmlrule

\section*{Making your first song}

This page assumes you have properly
installed \songlib and the initial samples. 
If you haven't yet,
please follow the directions in \see{install}.

The easiest way to write a \songlib program is to
base it upon an existing program. Here is
a working program that you can compile and,
in the future, modify.
These files are found in the quikstart directory
but are repeated here:

Name this file {\it quickstart.c}:

\begin{verbatim}
#include <stdio.h>
#include <stdlib.h>
#include <stdarg.h>
#include <songlib/util.h>
#include <songlib/songlib.h>

#define dir "/usr/local/share/samples/mandolin/"
#define base "sample_"

/* change PROGRAM_NAME and PROGRAM_VERSION appropriately */

char *PROGRAM_NAME = "quickstart";
char *PROGRAM_VERSION = "0.01";

static int last = 0;

static void
verse(int instrument,int octave)
    {
    startMeasure();

    //line 1

    chord(H,instrument,octave+1,C,-5,-12,(int) 0);
    chord(Q,instrument,octave+1,C,2,-5,-12,(int) 0);
    chord(Q,instrument,octave,C,7,14,(int) 0);

    chord(Q,instrument,octave+1,E,-9,-16,(int) 0);
    chord(H,instrument,octave,D,5,18,(int) 0);
    chord(Q,instrument,octave+1,F,-8,-15,(int) 0);

    chord(Q,instrument,octave+1,F,-8,-15,(int) 0);
    chord(H,instrument,octave,D,7,18,(int) 0);
    chord(Q,instrument,octave+1,F,-8,-15,(int) 0);

    chord(Q,instrument,octave+1,E,-9,-16,(int) 0);
    chord(Q,instrument,octave,D,5,-2,(int) 0);
    chord(H,instrument,octave+1,C,-5,-12,(int) 0);

    //line 2

    play(Q,instrument,octave,C);
    min(H,instrument,octave,G);
    play(H,instrument,octave,G);
    play(Q,instrument,octave,A);
    play(Q,instrument,octave,G);
    play(Q,instrument,octave,F);

    play(Hd,instrument,octave,G);
    chord(Q,instrument,octave+1,E,-9,-16,(int) 0);

    chord(Q,instrument,octave,C,7,16,(int) 0);
    chord(Q,instrument,octave+1,E,-9,-16,(int) 0);
    chord(Q,instrument,octave,D,-2,(int) 0);
    chord(Q,instrument,octave,C,7,12,(int) 0);

    chord(H,instrument,octave+1,C,-8,-12,(int) 0); //first half of next measure

    //line 3

    play(H,instrument,octave,C); //last half of measure

    play(Q,instrument,octave,G);
    play(Q,instrument,octave,F);
    play(H,instrument,octave,E);

    play(H,instrument,octave,F);
    chord(H,instrument,octave,G,-7,(int) 0);

    chord(H,instrument,octave,D,7,(int) 0);
    chord(Hd,instrument,octave,D,-3,(int) 0); //first quarter of next measure

    //line 4

    play(Q,instrument,octave,C); //last 3/4 of measure
    play(Q,instrument,octave,C);
    play(Q,instrument,octave,D);

    play(Q,instrument,octave,E);
    play(Q,instrument,octave,F);
    play(Q,instrument,octave,E);
    play(Q,instrument,octave,D);

    play(Q,instrument,octave,C);
    play(Q,instrument,1,A);
    chord(H,instrument,octave,F,-3,-7,(int) 0);

    chord(Q,instrument,octave,F,-3,-7,(int) 0);

    if (last) setSustain(0.999999);

    chord(Hd,instrument,octave,C,4,7,IX);

    checkMeasure();
    }

int
main()
    {
    int instrument;
    int octave = 2;

    songInit();

    instrument = readScale(dir,base);

    setTempo(250);
    setTime(4,4);
    setStride(0.05);
    setSustain(0.99995);
    setAmplitude(0.4);

    openOutput("quickstart.rra",0,0);

    verse(instrument,octave);
    last = 1;
    verse(instrument,octave);

    closeOutput();

    return 0;
    }
\end{verbatim}

Next, add a file named {\it makefile} to your working directory with the
following content:

\begin{verbatim}
INCLUDE=/usr/local/include/songlib/
LIB=/usr/local/lib/

# Place the names of all your tracks here (as RRA filenames)

RRAS = quickstart.rra

all : $(RRAS)

# comment out the rplay line if you don't want automatic playing

%.rra   : %.x
        ./$<

%.x     : %.c
		gcc -Wall -g -o $@ -I$(INCLUDE) $< -L$(LIB) -lsong -lm 

play    : $(RRAS)
        rrafastmixer -a0.5 $(RRAS) | rplay

.PHONY : clean

clean :
        -rm -f $(RRAS)
\end{verbatim}

{\bf IMPORTANT}:
Make sure all indented lines in your makefile begin with a tab character
and not spaces.

Finally, type:

\begin{verbatim}
    make
\end{verbatim}

and a \songlib\ program should be built and run, resulting
in an {\sc RRA} audio file named:

\begin{verbatim}
    quickstart.rra
\end{verbatim}

The {\sc RRA} can be played with this command:

\begin{verbatim}
    rplay quickstart.rra
\end{verbatim}

or this command:

\begin{verbatim}
    make play
\end{verbatim}

The latter plays the audio file at a reduced volume.

\section*{Decomposing quickstart.c}

For this discussion, focus your attention on the {\it main} function of
the {\it quickstart.c} file.

\subsection*{Initializing \songlib}

The {\it songInit} function readies the \songlib\ system for audio production.
It must be the first \songlib\ library function called and
is required.

\subsection*{Reading instrument notes}

The {\it readScale} function is used to read in a set of notes
in a given directory.
For example, the call:

\begin{verbatim}
    readScale("./","bright_");
\end{verbatim}

would read in all the {\sc RRA} files with prefix {\it bright\_} in the
current directory. These RRA files are then attached to the instrument
number that {\it readScale} returns.
For more information on the naming of {\it RRA} files that can be
read in with {\it readScale},
see \xlink{installing samples}{install.html}.
For more information on {\it readScale} and other methods for reading 
in {\sc RRA} files, 
see \xlink{reading samples}{readingSamples.html}.

\subsection*{Setting parameters}

Various settings of \songlib\ are controlled with function such as:

\begin{verbatim}
    setTempo(250);
    setTime(4,4);
    setStride(0.05);
    setSustain(0.99995);
    setAmplitude(0.4);
\end{verbatim}

For more information on these and other similar functions,
see \xlink{controlling output}{controlFunctions.html}
and \xlink{keeping time}{keepingTime.html}.

\subsection*{Designating the output}

\Songlib\ is capable of working with multiple sample rates and sample
sizes. Unless told otherwise, \songlib, however, assumes a sample rate
of 44100 samples per second and a sample size of 16 bits per sample.
One changes the sample rate and size globally with the {\it openOutput}
function. This ensures the output file will be written with the
given sample rate and size. As a side effect, if \songlib\ cannot
figure out the sample rate and size for a particular operation,
these given values will be used.

If the call to {\it openOutput} is passed zeroes for the sample rate and
size, as in:

\begin{verbatim}
    openOutput("track.rra",0,0);
\end{verbatim}

then the default values of 44100 samples per second and 16 bits per
sample are used. To set the output rra to 96000 samples per second
and 32 bits per sample, for example, the call would be:

\begin{verbatim}
    openOutput("track.rra",96000,32);
\end{verbatim}

\subsection*{Playing notes}

Between the calls to {\it openOutput} and {\it closeOutput},
one makes calls to
the various note playing functions in \songlib. The example program
seen previously makes a number of these calls, primarily to {\it play} and
{\it chord}. For more information on these note playing functions,
see 
\xlink{note playing}{notePlaying.html}.


\subsection*{Writing the audio file}

The last \songlib\ function to be called is {\it closeOuput}. This function
does the actual writing of the {\sc RRA} file specified in the call to
{\it openOutput}. This function is required.

\end{document}
