\documentclass{article}  
\input{header}

\htmltitle{Songlib: chords}

\title{Songlib: chords\\
\date{Revision Date: \today}}

\author{written by: Song Li Buser}


\begin{document}

\maketitle

\W\subsubsection*{\xlink{Printable Version}{chords.pdf}}
\W\htmlrule

\section*{Generating Chords}

In addition to the fundamental \see{chord} function,
\songlib\ provides some specialized chord generating
functions. The most powerful and easiest to use 
are the {\it modal} chords. Modal chords are explained in
the \seed{keySignatures}{key signatures}
and \see{modes} documents.

Although the use of modal chords is strongly encouraged, \songlib\
provides for a number of chord functions outside of the modal
system, described below.

\section*{Integer notation of pitches}

The classical naming of chords is based upon both
subjective and objective measures and is inherently
ambiguous. An alternate method of name chords is 
strictly objective and unambiguous. For example,
a {\it major} chord (regardless of key) is identified
by the {\it integer notation}:

\begin{verbatim}
    {4,7}
\end{verbatim}

meaning 
the second note in the chord is four semitones up from the 
base (or root) note, and the third note is seven
semitones up from the root.

The equivalent songlib chord using this notation
is:

\begin{verbatim}
    void i47(double beats,int instrument,int octave,int pitch)
\end{verbatim}

The root note of the chord is specified by the octave/pitch pair,
while the name specifies the second and third notes are four
and seven semitones above the root.
The call:

\begin{verbatim}
    i47(beats,instrument,octave,pitch);
\end{verbatim}

is therefore exactly equivalent to:

\begin{verbatim}
    chord(beats,instrument,octave,pitch,4,7,(int) 0);
\end{verbatim}

The advantage of using the {\it i}-chord functions is 
that they are only defined for nice-sounding chords,
that is chords that are commonly used in composition.
The other advantage is that it saves you a bit of typing.

The defined {\it i}-chords are:

\begin{center}
    {\it i3}, {\it i4}, {\it i5}, {\it i6}, {\it i7}, {\it i8}, and {\it i9}
\end{center}
\begin{center}
    {\it i37}, {\it i38}, {\it i47}, {\it i49}, {\it i57}, and {\it i59}
\end{center}

Each of these chords has a version suffixed by the
letter 'p', which indicates an additional note, one
octave above the root. Thus,

\begin{verbatim}
    i47p(beats,instrument,octave,pitch);
\end{verbatim}

...is equivalent to:

\begin{verbatim}
    chord(beats,instrument,octave,pitch,4,7,12,(int) 0);
\end{verbatim}

Keep in mind that music composed solely of nice sounding chords is often boring.

\end{document}
