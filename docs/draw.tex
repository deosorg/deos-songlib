\documentclass{article}  
\input{header}

\htmltitle{Songlib: draw}

\title{Songlib: draw\\
\date{Revision Date: \today}}

\author{John C. Lusth}


\begin{document}

\maketitle

\W\subsubsection*{\xlink{Printable Version}{draw.pdf}}
\W\htmlrule

The {\it draw} family of functions is used to smoothly play multiple notes
from a single sample. The intent is to play multiple notes as does
a violinist when changing the fingering on the neck of the
violin during a
a single draw of the bow.

\begin{verbatim}
void draw(double beats,int instrument,int octave,int pitch,double length,...,(int) 0);

void ndraw(double beats,int instrument,int numberedNote,double length,...,(int) 0);

void adraw(double beats,int instrument,int count,int *octaves,int * pitches,double *lengths);
\end{verbatim}

The {\it draw} function is a variadic function for playing multiple notes
from a single sample.
The sample is given by the instrument and the first triple
({\it instrument} and {\it octave},{\it pitch}, and {\it length}).
The variadic part contains
a series of ({\it octave},{\it pitch},{\it length}) triplets that
designate the subsequent notes and durations to be resampled from the
given sample. The variadic part is terminated by a zero.
Since subsequent notes 
start playing where the previous note left off in the
given sample, the transition does not produce any
discontinuites. 

The {\it draw} function is normally used when the notes have a distinctive
attack and the attack is not desired for short subsequent notes.

For example, this use of draw plays a little trill:

\begin{verbatim}
    draw(W,recorder,
        4,C,Q,
        4,D,Q,
        4,C,I,
        4,D,I,
        4,C,I,
        4,D,I,
        4,C,Q,
        (int) 0
        );
\end{verbatim}

The last triplet designates the pitch that is used of fill out the
total number of beats.
If there isn't enough data in the original sample, silence is played.

The {\it ndraw} function is similar to {\it draw} but takes numbered notes
instead of octave/pitch pairs. The variadic part
consists of {\it int numberedNote} and {\it double length}) pairs.

The example call to {\it draw} above could be equivalently rendered as:

\begin{verbatim}
    ndraw(recorder,
        C4,Q,
        D4,Q,
        C4,I,
        D4,I,
        C4,I,
        D4,I,
        C4,Q,
        (int) 0
        );
\end{verbatim}

Both {\it draw} and {\it ndraw} are wrapper functions for {\it adraw}.
The {\it adraw} function takes three parallel arrays of size {\it count}.
These arrays are filled with octave/pitch/length triplets.

There are also two convenient wrapper functions for {\it draw} when the goal
is to play two notes:

\begin{verbatim}
void resolve(int instrument,
    int firstOctave,int firstPitch,double firstBeats,
    int secondOctave,int secondPitch,double secondBeats);

void nresolve(int instrument,
    int firstNumberedNote,double firstBeats,
    int secondNumberedNote,double secondBeats);
\end{verbatim}

The following two calls are equivalent:

\begin{verbatim}
    resolve(W,violin,3,C,Q,3,D);
    draw(W,violin,3,D,0.0,3,C,Q,D3,W,(int) 0);
\end{verbatim}

\section*{Transitions between notes}

The transistions between notes within the {\it draw} family is controlled
by the variable {\it drawRamp}. The {\it drawRamp} setting specifies the number
of seconds it takes for a preceding note to slide into the following
note, using a linear ramp. The default setting of {\it drawRamp} is 0.25
seconds. You can get and set the {\it drawRamp} with the functions:

\begin{verbatim}
    double setDrawRamp(double seconds)
    double getDrawRamp(void)
\end{verbatim}

See also:
\xlink{playingNotes}{notePlaying.html},
\end{document}
