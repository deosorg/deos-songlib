\documentclass{article}  
\input{header}

\newcommand\ver {{1.47}}

\htmltitle{Songlib: The C Library for Music Composition}

\title{Songlib: The C Library for Music Composition\\
\date{Revision Date: \today}}

\author{written by: John C. Lusth}

\begin{document}

\maketitle

\W\subsubsection*{\xlink{Printable Version}{cygwin-install.pdf}}
\W\htmlrule

\section*{Cygwin/Windows installation instructions}

First, browse to
\xlink{www.cygwin.com}{http://www.cygwin.com} and download the current
{\it setup.exe} and run it. Use the 64-bit version of the setup program if 
you have a 64-bit computer.

Answer all the questions using the default answers. At some point,
you will be asked to select some packages. Click on plus signs next
to

\begin{itemize}
\item
    'Audio' and click on the Skip column entry next to the {\it sox} entry.
\item
    'Devel' and click on the Skip column entry next to the {\it gcc-core}
    and {\it make} entries.
\item
    'Editors'
    and click on the Skip labels next to the {\it vim} and {\it vim-common}
    entries.
\item
    'Net'
    and click on the Skip label next to the {\it openssh} entry.
\item
    'Util'
    and click on the Skip label next to the {\it ncurses} entry.
\item
    'Web'
    and click on the Skip label next to the {\it wget} entry.
\end{itemize}

Finally, click on the {\it Next} button in the lower right corner
and continue accepting defaults.

Once Cygwin is installed, open up a Cygwin terminal (an icon should
be on your Desktop).
Begin the songlib installation process by downloading the \songlib\ source
tarball into a directory named {\it songlib}:

\begin{verbatim}
    mkdir ~/songlib
    cd ~/songlib
    wget beastie.cs.ua.edu/songlib/songlib-1.47.tgz
\end{verbatim}

Then, unpack the tarball:

\begin{verbatim}
    tar xvfz *.tgz
\end{verbatim}

Finally, build and install the library and install the utilities
and sample note files by typing the command:

\begin{verbatim}
    make install
\end{verbatim}

while in the {\it songlib} directory.

To test your installation, move into the {\it quickstart} directory and
type {\it make play}:

\begin{verbatim}
    cd quickstart
    make play
\end{verbatim}

You should see the song being constructed and then should hear it
begin to play.

\end{document}
