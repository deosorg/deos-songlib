\documentclass{article}  
\input{header}

\htmltitle{rra: readily readable audio}

\title{RRA: readily readable audio\\
\date{Revision Date: \today}}

\author{John C. Lusth}

\begin{document}

\maketitle

\W\subsubsection*{\xlink{Printable Version}{rra.pdf}}
\W\htmlrule

\section{Introduction}

The purpose of this document is to specify the audio format
used by {\it songlib}. The RRA (Readily Readable Audio) format is designed
to be easily viewed (and debugged) and extensible. Originally
suggested by Ian Taylor for use in a High School Programming
Contest, it has been adopted as the fundamental format by {\it songlib}
to allow for less-experienced programmers to improve and extend 
the library.

An RRA file is composed of two sections, a header section and
a data section.

\section{Grammar}

An RRA file has the following grammar. Non-terminals are in
lowercase and terminals are in UPPERCASE. An RRA file begins
with the seven ASCII characters RRAUDIO.

\begin{verbatim}
    file : "RRAUDIO" optTags "%%" data

    optTags : *empty*
            | tag optTags

    tag : attribute ":" value

    value : IDENTIFIER | INTEGER | STRING

    data : INTEGER
         | INTEGER data
\end{verbatim}

An RRA file is free-format and whitespace delimited.
Comments begin with a bang (!) and continue until the
end of line.  Here is an example file:

\begin{verbatim}
    RRAUDIO
    ! Hohner Pocket Pal Harmonica
    ! D#
    ! recorded by Becky Smith, February 2011
    sampleRate: 44100
    bitsPerSample: 24
    channels: 1
    replicates: "1203928,2593821"
    %%
    ! amplitude data begins
    0
    0
    1
    -2
    ...
\end{verbatim}

Note that blank lines, whitespace before a colon, and
tags spanning multiple lines are all permissible. Comments
may appear in the data section as well.

The token '\%\%' separates the header from the data section
of an RRA file.

\section{Required tags}

No tags are required but all applications must support the following tags:

\begin{verbatim}
    channels: <positive integer>
    sampleRate: <positive integer>
    bitsPerSample: <postive integer>
    samples: <a non-negative integer>
    skip: <a non-negative integer>
\end{verbatim}

If not present, the following values are assumed for the
the supported attributes:

\begin{verbatim}
    channels: 1
    sampleRate: 44100
    bitsPerSample: 16
    samples: 0
    skip: 0
\end{verbatim}

The value for {\it samples} indicates the number of samples
in one channel. To find the total number of samples
across all channels, one would multiply {\it samples} by
{\it channels}.

A zero value for the {\it samples} attribute indicates that
the number of samples is not specified and an application
should continue to read data until the end-of-file. If at
all possible, an application should set the {\it samples} attribute
accurately. It is up to the application to decide how
to handle a {\it samples} value that doesn't match the number
actual amount of amplitude data. At a minimum, a warning
message should be generated.

The {\it skip} attribute indicates how many samples at the beginning
of the audio stream should be ignored by the application.

\section{Duplicate tags}

The meaning of duplicate tags is not specified. Three reasonable
alternatives come to mind:

\begin{itemize}
\item
    all subsequent tags are ignored
\item
    all previous tags are ignored
\item
    the attribute is variadic and the values accumulate
\end{itemize}

For the first two cases, duplicate warning messages should be
generated.

Instead of using duplicate tags for multiple values, one
could also use a string to encapsulate multiple values.
For example, the {\it replicates} tag might be specified
thusly:

\begin{verbatim}
    replicates : "87231,169842,27154"
\end{verbatim}

...meaning that while the note starts at sample 0, the first replicate
starts at sample 87231, the second at 169,842, and so on.
It is up to the application to decide how multiple values are
to be communicated.

\section{Unknown tags}

An application should handle unknown tags gracefully. At a mininum,
an unknown tag should be ignored but a warning message should
be generated.

\section{Multiple channels}

The number of channels specifies how many different audio streams are
combined to make the whole. For example, monaural audio (channels: 1)
has a single channel.  Stereo audio (channels: 2) has two channels
(left and right), while quadraphonic audio (channels: 4) has four
channels, and so on. The amplitude data for multiple channels
are interleaved. For stereo, the amplitude data/samples looks like this:

\begin{verbatim}
    Amplitude_0_Channel_0
    Amplitude_0_Channel_1
    Amplitude_1_Channel_0
    Amplitude_1_Channel_1
    Amplitude_2_Channel_0
    Amplitude_2_Channel_1
    ...
    Amplitude_N-1_Channel_0
    Amplitude_N-1_Channel_1
\end{verbatim}

Channels are numbered starting at zero.

A warning message should be generated if the total number of
samples is not evenly divisible by the number of channels.

\section{Warning messages}

An application should allow the user to turn off warning messages.

\end{document}
