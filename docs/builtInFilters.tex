\documentclass{article}  
\input{header}

\htmltitle{Songlib: built-in filters}

\title{Songlib: built-in filters
\date{Revision Date: \today}}

\author{Song Li Buser}

\date{Revision Date: \today}

\begin{document}

\maketitle

\W\subsubsection*{\xlink{Printable Version}{builtInFilters.pdf}}
\W\htmlrule

The following filters come with the {\it\bf songlib} system. For information
on how to use them, see
\see{filters}.

\htmlrule

Funtion: 

\begin{verbatim}
void lowPass(int *data,int length,double frequency,double resonance);
\end{verbatim}

This filter removes the higher frequencies from the audio data.
Frequencies above the given \emph{frequency} are severely attenuated.
According to the
\xlink
{http://www.musicdsp.org}
{http://www.musicdsp.org}
website, the \emph{resonance}
parameter should be between $\mathrm{sqrt}(2)$ (lowest resonance)
to 0.1 (highest resonance),
assuming \emph{frequency} is greater than three or four
kilohertz. For more info on acoustic resonance, see
\xlink
{http://en.wikipedia.org/wiki/Acoustic\_resonance}
{http://en.wikipedia.org/wiki/Acoustic\_resonance}.

4000 : the 1kHz - 4kHz mid frequency band is where the human ear is most
sensitive

1.414 : the square root of 2 (1.414) is the ratio between the average and
peak values of a sine wave

\htmlrule

Function:

\begin{verbatim}
   void highPass(int *data,int length,double frequency,double resonance);
\end{verbatim}

Like \emph{lowPass} but attenuates frequencies below the given \emph{frequency}.

\htmlrule

Function:

\begin{verbatim}
    void amplify(int *data,int length,double amp);
\end{verbatim}

This filters scales all the values in \emph{data} by \emph{amp}. If \emph{amp}
is greater than one, the effect will be to increase the volume.
Conversely, a value less than one will decrease the volume.

\htmlrule

Function:

\begin{verbatim}
    void attackLinear(int *data,int length,double amp,double delta)
\color{black}

There are two important cases for this filter. The first case is:

\begin{verbatim}
    amp < 1 and delta > 0
\end{verbatim}

This softens the first part of the note. Sample i is scaled thusly:

\begin{verbatim}
    if (amp + delta * i < 1)
	i = i * (amp + delta * i);
\end{verbatim}

The first sample is scaled the most, the second a little less, the
third, a little less yet, and so on.

The other important case is:

\begin{verbatim}
    amp > 1 and delta < 0
\end{verbatim}

This increases the loundess of the first part of the note.
Sample \emph{i} is scaled thusly:

\begin{verbatim}
    if (amp + delta * i > 1)
	i = i * (amp + delta * i);
\end{verbatim}

Here are two typical calls:

\begin{verbatim}
    attackLinear(data,length,0.5,0.0002);

    attackLinear(data,length,1.5,-0.0002);
\end{verbatim}

\htmlrule

Function:

\begin{verbatim}
    void attackExponential(int *data,int length,double amp,double delta);
\end{verbatim}

Like \emph{attackLinear} only the ramp is exponential.
Sample \emph{i} is scaled
as follows:

\begin{verbatim}
    i = i * amp * pow(delta,i);
\end{verbatim}

Here are two typical calls:

\begin{verbatim}
    attackExponential(data,length,0.5,0.100075);

    attackExponential(data,length,1.5,0.99995);
\end{verbatim}

\htmlrule

Function:

\begin{verbatim}
    void diminishLinear(int *data,int length,int offset,double delta);

    void diminishExponential(int *data,int length,int offset,double factor);
\end{verbatim}

Like the attack filters, but works on the end of the data rather than
the beginning. Sample \emph{i} is updated thusly:

\begin{verbatim}
    i = i * (1 + delta * i);  //linear

    i = i * pow(factor,i);    //exponential
\end{verbatim}

\htmlrule

Function:

\begin{verbatim}
    distort1(int *data,int length,int cutoff);

    distort2(int *data,int length,int cutoff);

    distort3(int *data,int length,int cutoff,double level);
\end{verbatim}

Three filters for adding distortion.
\end{document}
