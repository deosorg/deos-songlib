\documentclass{article}  
\input{header}

\htmltitle{songlib: modes}

\title{Songlib: modes\\
\date{Revision Date: \today}}

\author{written by: Song Li Buser}

\begin{document}

\maketitle

\W\subsubsection*{\xlink{Printable Version}{modes.pdf}}
\W\htmlrule

\section*{Musical modes}

Musical modes provide a structure for generating
nice sounding chords. There are seven modern
musical modes:

\begin{itemize}
\item Ionian
\item Dorian
\item Phrygian
\item Lydian
\item Mixolydian
\item Aeolian
\item Locrian
\end{itemize}

The {\it Ionian}, {\it Lydian}, and {\it Mixolydian} modes are considered
major modes and have a brighter and more cheery 
feel than the other listed modes, which are considered
minor modes.

To read more about modes, see
\xlink{Musical Modes}{http://en.wikipedia.org/wiki/Musical_mode}
on Wikipedia.

\section*{Using modal chords}

It is not actually necessary to understand modes
to use them as a composition tool in \songlib.
By setting a particular mode, \songlib\ allows
you to generate {\it modal} chords compatible with that mode quite easily.
The trichord generating function
is named {\it c}, with the first argument specifying
the chord number:

\verb!c(1,...)!, \verb!c(2,...)!, \verb!c(3,...)!, \verb!c(4,...)!,
    \verb!c(5,...)!, \verb!c(6,...)!, and \verb!c(7,...)!

A trichord is simply a chord of three notes.
If you are building a song in the key of C, then the
\verb!c(1,...)! chord has a base (or {\it root}) note of C,
the \verb!c(2,...)! chord has a root
note of D, and so on through E, F, G, A, and B. If you
are using the key of E, the \verb!c(1,...)! chord has a root note
of E, the \verb!c(2,...)! chord has a root note of F\#, and so on.

Thus, an {\it Ionian} riff in the key of {\it D} might look like the
following in \songlib:

\begin{verbatim}
    setKey(D);
    setMode(IONIAN);

    for (i = 0; i < 4; ++i)
        {
        c(1,H,guitar,octave);
        c(4,H,guitar,octave-1);
        c(5,H,guitar,octave-1);
        c(5,H,guitar,octave-1);
        c(4,Hd,guitar,octave-1);
        }
\end{verbatim}

For simplicity, we will sometimes abbreviate the call \verb!c(1,...)! as 
{\it c1}, \verb!c(2,...)! as {\it c2}, and so on.

Notice the absence of F\# in the above riff. This is because \songlib\ figures
out the root note and the other notes in the chord automatically when the
{\it c} function is used.

\section*{Chord progressions}

When using the modal system, not only does \songlib\ figure out which notes go
with which chord, you also get, for free, a bullet-proof way to compose songs using chord progressions.
A chord progression is simply two or more chords in a row. Some very smart
people have figured out which chords sound `nice' next to each other. Here
are the general purpose charts for designing your own nice-sounding chord
progressions.

For the {\it Ionian} or major mode, the chord progression generating chart looks
like:

\begin{itemize}
\item
        {\it c3}
\item
        {\it c6}
\item
        {\it c2} or {\it c4}
\item
        {\it c5} (then go to {\it c1}) or {\it c7} (then go to {\it c3})
\item
        {\it c1}
\item
        any chord
\end{itemize}

To generate a chord progression, you can start at any bullet.
Say we start at {\it c1}. Then we go to the next bullet. It says
any chord, so we can go to any bullet. Let's go to the
{\it c2} or {\it c4}
bullet and choose {\it c4}. Again, we go to the next bullet 
and choose {\it c5}. Finally, we return to {\it c1}.
It is common for major chord progressions to end at {\it c1}.
Our little journey yielded the progression:

\begin{quote}
    {\it c1} $\rightarrow$ {\it c4} $\rightarrow$ {\it c5} $\rightarrow$ {\it c1}
\end{quote}

For the {\it harmonic minor} mode,
the generating chart is slightly different:

\begin{itemize}
\item
        {\it cM7}
\item
        {\it c3}
\item
        {\it c6}
\item
        {\it c2} (then go to {\it c5}
            or {\it c7}) or {\it c4} (then go to {\it c7})
\item
        {\it c5} or {\it c7}
\item
        {\it c1}
\item
        any chord
\end{itemize}

Note that this chart introduces a new chord, {\it cM7}, which stands
for a {\it Major c7}. In the harmonic minor mode, {\it c7} is a a diminished chord,
so this chart features two versions of {\it c7}, one major and one diminished.
The next section covers forcing chords in one mode to behave as if in
another mode.

\section*{Playing chords out of mode}

No matter the mode, some of the seven possible chords are major
chords, some are minor, some are augmented, and some are diminished chords.
\Songlib\ figures this out automatically, so composing using the
modal system is quite easy.
However, the modal system described above is a little
too confining. For example, in {\it Ionian} mode, {\it c1} is a major chord.
But perhaps you wish
to throw in a minor version of {\it c1}.
To do so, simply call the \verb!cm(1,...)! function
instead of \verb!c(1,...)!.

\begin{verbatim}
    setKey(D);
    setMode(IONIAN);

    for (i = 0; i < 4; ++i)
        {
        cm(1,H,guitar,octave);
        c(4,H,guitar,octave-1);
        c(5,H,guitar,octave-1);
        c(5,H,guitar,octave-1);
        c(4,Hd,guitar,octave-1);
        }
\end{verbatim}

In {\it Ionian} mode,
the three major chords are {\it c1}, {\it c4}, and {\it c5},
and the three minor chords are {\it c2}, {\it c3}, and {\it c6}.
The last chord, {\it c7}, is
a diminished chord.
In {\it Aeolian} mode, the major chords are {\it c3}, {\it c6}, and {\it c7},
while {\it c1}, {\it c4}, and
{\it c5} are minor chords and {\it c2} is a diminished chord.
All chords have major, minor, augmented, and
diminished versions, obtained by appending the appropriate
letter to the chord name; the major, minor, augmented,
and diminished versions are named, {\it cM}, {\it cm}, {\it ca}, and {\it cd},
respectively.

Thus for the progression generating chart for the {\it harmonic minor} mode,
shown in
the previous section, the {\it cM7} placeholder would be implemented by
a call to \verb!cM(7,...)!.
Since {\it c7} is a diminished chord in the {\it harmonic minor} mode,
it could be implemented as:

\begin{verbatim}
    chord(beats,instrument,octave,C,+3,+6,(int) 0);
\end{verbatim}

A diminished chord lowers both the
second note and the third note one semitone each.
A call to the {\it harmonic minor} chord \verb!cM(7,...)! in the key of C is 
equivalent to the following call to {\it chord}:

\begin{verbatim}
    chord(beats,instrument,octave,C,+4,+7,(int) 0);
\end{verbatim}

Augmenting a chord generally means to
raise the third note in the chord one semitone. For example,
the {\it Aeolian} {\it c7} chord is a major chord, so
calling \verb!ca(7,...)! in that mode in the key of C is equivalent to:

\begin{verbatim}
    chord(beats,instrument,octave,C,+4,+8,(int) 0);
\end{verbatim}

\section*{Degrees}

In music theory, the degree of a note indicates
its index with respect to the root note in the
scale. The modal system of \songlib\ uses a
diatonic scale having seven notes. In the key
of {\it C}, the index of {\it C} is 1, {\it D} is 2, {\it E} is 3 and
so on. In the key of {\it D}, the index of {\it D} is 1,
{\it E} is 2, {\it F\#} is 3 and so on. 

The degree function returns the root note of the associated modal
chord. For example,
if you are in the key of {\it C}, then
{\tt degree(1)} returns a {\it C},
{\tt degree(2)} returns a {\it D},
and so on.

Since {\it c1} is a major chord in {\it Ionian} mode, it is therefore 
equivalent, under this scenario, to all of the following calls:

\begin{verbatim}
    chord(H,guitar,octave,C,+4,+7,(int) 0);
    chord(H,guitar,octave,degree(1),+4,+7,(int) 0);
    maj(H,guitar,octave,C);
    maj(H,guitar,octave,degree(1));
\end{verbatim}

when in the key of C.
Likewise, {\it c2} is a minor chord and
is thus equivalent to all these calls:

\begin{verbatim}
    chord(H,guitar,octave,D,+3,+7,(int) 0);
    chord(H,guitar,octave,degree(2),+3,+7,(int) 0);
    min(H,guitar,octave,D);
    min(H,guitar,octave,degree(2));
\end{verbatim}

Of course, you can always use {\it chord} and hardwire the notes,
but then it is difficult to change the key of your composition.
By using the degree {\it function} with {\it chord}, it is easy to change the
key,
but you have to keep track of which
chords sound appropriate, i.e. when to play major and
when to play minor chords. \Songlib's modal system
takes care of all these problems automatically.
    
\section*{Modes and their chords}

Note that
every mode uses some combination of major and
minor chords in its scale. For example, 
the {\it Ionian} mode actually uses the chords:

\begin{center}
{\it maj/{\tt degree(1)}},
{\it min/{\tt degree(2)}},
{\it min/{\tt degree(3)}},
{\it maj/{\tt degree(4)}},
{\it maj/{\tt degree(5)}},
{\it min/{\tt degree(6)}}, and
{\it dim/{\tt degree(7)}}
\end{center}

so in {\it Ionian} mode, {\it c1} is equivalent to {\it maj/degree(1)},
and so on.

The following table lists the types of chords used for
every mode:

\begin{center}
\begin{tabular}{lccccccc}
\T\toprule
\texonly{\multirow{2}{*}{{\it Mode}} & \multicolumn{7}{c}{Chord Number} \\}
\htmlonly{ {\it Mode} & & & {\it Chord} & {\it #} & & & \\}
%\cmidrule{2-8}
               & 1   & 2   & 3   & 4   & 5   & 6   & 7   \\
\T\midrule
Ionian         & maj & min & min & maj & maj & min & dim \\
Dorian         & min & min & maj & maj & min & dim & maj \\
Phrygian       & min & maj & maj & min & dim & maj & min \\
Lydian         & maj & maj & min & dim & maj & min & min \\
Mixolydian     & maj & min & dim & maj & min & min & maj \\
Aeolian        & min & dim & maj & min & min & maj & maj \\
Locrian        & dim & maj & min & min & maj & maj & min \\
Melodic Minor  & min & min & aug & maj & maj & dim & min \\
Harmonic Minor & min & dim & aug & min & maj & maj & dim \\
\T\bottomrule
\end{tabular}
\end{center}

\W ~\newline

To use {\it Aeolian} mode, for example, you could use
the chords:

\begin{center}
{\it min}/{\tt degree(1)},
{\it dim}/{\tt degree(2)},
{\it maj}/{\tt degree(3)},
{\it min}/{\tt degree(4)},
{\it min}/{\tt degree(5)},
{\it maj}/{\tt degree(6)},
and {\it maj}/{\tt degree(7)}
\end{center}

Of course, \songlib\ figures out which chords and degrees to use if
you stick with calls to \verb!c(1,...)! through \verb!c(7,...)!.

Note the addition of two more modes in the table, the {\it Melodic}
and {\it Harmonic} {\it Minors}.
Furthermore, there are three alias modes, {\it Major}
(which is the same as {\it Ionian}) and {\it Minor} and
{\it Natural} {\it Minor}
(which are aliases for {\it Aeolian}).

The {\it setMode} function is used to set the mode. To
set the desired mode, capitalize the mode name and
substitute underscores for spaces. For example,
to set the {\it Harmonic} {\it Minor} mode, make this call:

\begin{verbatim}
    setMode(HARMONIC_MINOR);
\end{verbatim}

To change the mode within a song, simply call {\it setMode} with the
new mode. Any chords generated subsequently will use the new
mode.

\section*{Flattened and sharpened chords}

Sometimes, it is desirable to call a chord in the next
lower (or next higher) key. One can {\it flatten} and {\it sharpen} the
modal chords for this task. The chords are obtained by calling the
{\it cf} function
(for flatten)
and the {\it cs} function (for sharpen) to the modal chord name.
For example, suppose your
\songlib\ program is in E-{\it Phrygian}:

\begin{verbatim}
    setKey(E);
    setMode(PHRYGIAN);
\end{verbatim}

and you wish to play a {\it c3} chord in the key of {\it D\#}.
You would make a call similar to this:

\begin{verbatim}
    cf(3,beats,instrument,octave);
\end{verbatim}

To play a {\it c4} chord in the key of {\it F}, the call would
look similar to:

\begin{verbatim}
    cs(4,beats,instrument,octave);
\end{verbatim}

\section*{Seventh chords}

The \songlib\ library also has functions for generating
modal seventh chords. A seventh chord has an additional
note one third higher than the highest note
in a trichord, yielding a total of four notes, although,
sometimes, the next to the highest note in the chord is omitted.

The seventh chord functions are exactly the same
as the basic modal
trichord functions, but the {\it c7} function is called instead
of the {\it c} function.
To provide more control over seventh chords,
\songlib\ provides these additional functions:

\begin{center}
\begin{tabular}{lcccr}
\T\toprule
{\it Common name} & {\it technical name} & {\it modal function} 
    & {\it semitone deltas}\\
\T\midrule
Major $7^{th}$ & Major/Major & cM7 & {\tt 4, 7, 11}\\
Minor $7^{th}$ & Minor/Minor & cm7 & {\tt 3, 7, 10}\\
Augmented $7^{th}$ & Augmented & ca7 & {\tt 4, 8, 11}\\
Diminished $7^{th}$ & Diminished/Diminished & cd7 & {\tt 3, 6, ~9}\\
Dominant $7^{th}$ & Major/Minor & cD7 & {\tt 4, 7, 10}\\
\T\bottomrule
\end{tabular}
\end{center}

\W ~\newline

Recall that the degree function can be used to automatically derive the
correct pitch. In {\it Ionian} mode, the seventh chord with a root of degree 3
would be a minor chord. Thus, in the non-modal system, one would
call:

\begin{verbatim}
    chord(beats,instrument,octave,degree(3),+3,+7,+10,(int) 0);
\end{verbatim}

Using the modal system, the call would be:

\begin{verbatim}
    c7(3,beats,instrument,octave);
\end{verbatim}

in {\it Ionian} mode and 

\begin{verbatim}
    cm7(3,beats,instrument,octave);
\end{verbatim}

in modes where {\it c3} is not a minor chord.
    
If the current key is the key of {\it C} and the mode is {\it Ionian}, the
notes in both the above chords would be {\it E}, {\it G}, {\it B},
and {\it D}.
Note that there
are three semitones between {\it E} and {\it G}, seven semitones between
{\it E} and {\it B}, and ten semitones between {\it E} and {\it D}.

\section*{Extended chords}

The \songlib\ system provides, ninth, eleventh, and thirteenth
chords as well as seventh chords. Those functions are
{\it c9}, {\it c11}, and {\it c13}, respectively.
One can also flatten and sharpen these chords with the functions
{\it c9f},
{\it c9s},
{\it c11f},
{\it c11s},
{\it c13f},
and
{\it c13s}.

With ninth chords, \songlib\ omits the $3^{rd}$ highest note. With
eleventh chords, the $4^{th}$ and $2^{nd}$ highest notes are omitted, while
with thirteenth chords, the $5^{th}$, $3^{rd}$, and $2^{nd}$ highest notes
are omitted.
The next section discusses
how to establish finer control over the modal system,
including the omission of notes.

\section*{Fine control of the modal chord system}

All the modal chords are based upon a general purpose function
named {\it cchord}. If you wish to control
the modal chord system at a lower level,
you can call the {\it cchord} function directly.
The function
has the following signature:

\begin{verbatim}
    cchord(chordNumber,duration,instrument,octave,tweaks);
\end{verbatim}

where
{\it chordNumber} is selected from the set \{1,2,3,4,5,6,7\} and
and {\it tweaks} is an string of adjustments the sharpen,flatten,
or omit
the individual notes in the chord. For example, the {\it c} chord
is simply a wrapper to the call:

\begin{verbatim}
    cchord(chordNumber,duration,instrument,octave,"xxx");
\end{verbatim}

In the tweaks string, the following encoding
is used:

\begin{tabular}{cl}%
\T\toprule
{\it character} & {\it meaning} \\
\T\midrule
{\it -}      & corresponding note is omitted \\
{\it x}      & corresponding note is left alone \\
{\it X}      & corresponding note is raised one octave \\
{\it y}      & same as {\it x} \\
{\it Y}      & corresponding note is lowered one octave \\
{\it n}      & corresponding note is the successor note in the scale/mode \\
{\it N}      & line {\it n}, but one octave higher \\
{\it m}      & like {\it n}, but two notes up \\
{\it m}      & like {\it m}, but one octave higher \\
{\it l}      & like {\it n}, but three notes up \\
{\it L}      & like {\it l}, but one octave higher \\
{\it p}      & corresponding note is the predecessor note in the scale/mode \\
{\it P}      & like {\it p}, but one octave down \\
{\it q}      & like {\it p}, but two notes down \\
{\it Q}      & like {\it q}, but one octave lower \\
{\it r}      & like {\it p}, but three notes down \\
{\it R}      & like {\it r}, but one octave lower \\
{\it s}      & corresponding note is adjusted up one semitone \\
{\it S}      & corresponding note is adjusted up two semitones \\
{\it u}      & corresponding note is adjusted up three semitone \\
{\it U}      & corresponding note is adjusted up four semitones \\
{\it b}      & corresponding note is adjusted down one semitone \\
{\it B}      & corresponding note is adjusted down two semitones \\
{\it d}      & corresponding note is adjusted down three semitone \\
{\it D}      & corresponding note is adjusted down four semitones \\
\T\bottomrule
\end{tabular}

Thus, the chord {\it c7} is a wrapper to the call:

\begin{verbatim}
    cchord(chordNumber,duration,instrument,octave,"xxxx");
\end{verbatim}

If {\it c4} is a major chord,
then it can be tweaked into a dominant seventh chord
using the following call to {\it cchord}:

\begin{verbatim}
    cchord(4,duration,instrument,octave,"xxxp");
\end{verbatim}

which flattens the seventh note one step down the scale.
If {\it c4} is a minor chord,
it can be tweaked into an augmented seventh chord with the call:

\begin{verbatim}
    cchord(4,duration,instrument,octave,"xnnx");
\end{verbatim}

sharpening the middle two notes of the chord.

One should generally stick with tweaks from the set:

\begin{verbatim}
    X Y n N m M l L p P q Q r R
\end{verbatim}

Applying a pleasing sounding tweak using semitone shifts
in one mode will
likely sound unpleasant if the mode or key is changed.

In general, the {\it cchord} function is called directly if you
wish to change the notes that are omitted in an extended chord.
For example, a {\it c9} chord is a wrapper to:

\begin{verbatim}
    cchord(chordNumber,duration,instrument,octave,"xx-xx");
\end{verbatim}

This tweaks string omits the middle note of the chord; this note
is known as {\it the fifth}.
Music theorists use what seems, at first, a strange terminology
for naming the notes of the chord. The following table gives
the note names that correspond to the position in the tweaks string:

\begin{tabular}{cl}%
\T\toprule
{\it string position} & {\it note name} \\
\T\midrule
1           & the root\\
2           & the third \\
3           & the fifth \\
4           & the seventh \\
5           & the ninth \\
6           & the eleventh \\
7           & the thirteenth \\
\T\bottomrule
\end{tabular}

If you wish to omit the third instead of the fifth, you would call
{\it cchord} like this:

\begin{verbatim}
    cchord(chordNumber,duration,instrument,octave,"x-xxx");
\end{verbatim}

\section*{Other modal chords}

There are a few other modal chord functions of note:

\begin{tabular}{cl}%
\T\toprule
{\it function} & {\it result} \\
\T\midrule
\verb!ic!           & for example, ic(5,2,beats,inst,octave) which plays the lower two notes of chord 5 an
    octave higher -- the larger the second argument, the more notes are inverted(raised an octave)\\
\verb!cpower!           &  plays a power chord; the middle note is replaced by the root note at one higher octave\\
\verb!cpower2!           &  like {\it cpower}, but the middle note is played as well\\
\verb!csus2!           &  plays a suspended chord; the middle note is replaced by the major second\\
\verb!csus4!           &  plays a suspended chord; the middle note is replaced by the perfect fourth\\
\verb!cquart!           &  plays a quartal chord (stacked fourths)\\
\verb!b!           &  plays  a broken chord (arpeggio)\\
\T\bottomrule
\end{tabular}

All these chords have the same arguments as {\it c}, except the broken chord
function {\it b}. With a broken chord, the individual notes are played
sequentially or in parallel, according to a set of given patterns, rather than
all in parallel. For example, the call:

\begin{verbatim}
    b(1,H,piano,octave,"-x-","x-x",(char *) 0);
\end{verbatim}

would play the middle note of the chord followed by the root and the
top note played simultaneously. The duration of each pattern is the
duration of the chord divided by the
number of patterns. In the above example, each pattern is played
for a quarter note.
Note that tweaks can be used in a broken chord pattern. Also, an
``unbroken'' broken chord is equivalent to {\it cchord}. The following
calls are equivalent:

\begin{verbatim}
    b(3,W,piano,octave,"xxx",(char *) 0);
    cchord(3,W,piano,octave,"xxx");
    c(3,W,piano,octave);
\end{verbatim}

You can adjust the amplitude each pattern in a broken chord with a call
to {\it setBrokenAmplitudes}:

\begin{verbatim}
    setBrokenAmplitudes(1.0,0.5,DX);
    b(1,H,piano,octave,"-x-","x-x",(char *) 0);
\end{verbatim}

Each argument to {\it setBrokenAmplitudes} is a factor with which the current
amplitude is multiplied before the pattern is played. In this example,
the first pattern \verb!"-x-"! is played at the current amplitude while the
second pattern \verb!"x-x"! is played at half the current amplitude.
The factors sent to {\it setBrokenAmplitudes} persist until the next call to
{\it setBrokenAmplitudes}.

\section*{Playing single notes in the modal system}

You can play the individual notes in the current key, consistent with
the current mode. These single note playing functions have the form:

\begin{verbatim}
    n(chordNumber,duration,instrument,octave,offset)
\end{verbatim}

The offset is chosen from the set \{0..6\}. These numbers
are the equivalent of {\it do}, {\it re}, {\it mi}, {\it fa}, {\it so},
{\it la}, and {\it ti}. For example, the {\it c1} function is equivalent
to playing the following notes simultaneously:

\begin{verbatim}
    n(1,H,piano,octave,0);
    n(1,H,piano,octave,2);
    n(1,H,piano,octave,4);
\end{verbatim}

You can flatten a note by calling {\it nf} instead of {\it n}.
Likewise, you can sharpen a note by calling the {\it ns} function.
For example,
the chord {\it cm} is equivalent to the following notes played simultaneously:

\begin{verbatim}
    n(1,H,piano,octave,0);
    nf(1,H,piano,octave,2);
    n(1,H,piano,octave,4);
\end{verbatim}

A power chord omits the third and adds in the root note at
an octave higher.
You can play a power chord by combining calls to {\it cchord} and {\it n}.
Here is one form of a power chord:

\begin{verbatim}
    cchord(chordNumber,duration,instrument,octave,"x-x");
    backwards(duration);
    n(chordNumber,duration,instrument,octave+1,0);
\end{verbatim}

This form has been encapsulated in the {\it cpower} function,

Finally, you can retrieve
the actual note in a modal chord by using the {\it getcnote}
function. These two calls are equivalent:

\begin{verbatim}
    n(3,H,piano,5);
    play(H,piano,getcnote(3,5));
\end{verbatim}

\end{document}
