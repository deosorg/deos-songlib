\documentclass{article}  
\input{header}

\htmltitle{songlib: rra support}

\title{Songlib: rra support\\
\date{Revision Date: \today}}

\author{written by: Song Li Buser}

\begin{document}

\maketitle

\W\subsubsection*{\xlink{Printable Version}{rraSupport.pdf}}
\W\htmlrule

\section*{A sample program}

Here is a typical program for manipulating an RRA file. This program
clips the audio so that no value exceeds the preset level:

\begin{verbatim}
#include <stdio.h>
#include <stdlib.h>
#include <stdarg.h>
#include <string.h>
#include <math.h>
#include <assert.h>
#include <songlib/util.h>
#include <songlib/rra.h>

#define CLIP_LEVEL 3000

/* change PROGRAM_NAME and PROGRAM_VERSION appropriately */

char *PROGRAM_NAME = "clip";
char *PROGRAM_VERSION = "0.01";

static int clip = CLIP_LEVEL;
static int amplify = 32768 / CLIP_LEVEL;

static int processOptions(int,char **);

int
main(int argc,char **argv)
    {
    int argIndex = 1;

    int i,j;
    int amp;
    FILE *in,*out;
    RRA *h;

    argIndex = processOptions(argc,argv);

    if (argc-argIndex == 0)
        {
        in = stdin;
        out = stdout;
        }
    else if (argc-argIndex == 1)
        {
        in = OpenFile(argv[argIndex],"r");
        out = stdout;
        }
    else if (argc-argIndex == 2)
        {
        in = OpenFile(argv[argIndex],"r");
        out = OpenFile(argv[argIndex+1],"w");
        }
    else
        {
        printf("usage: clip -aN -cN [<input rra file> [<output rra file>]]\n");
        exit(-1);
        }

    h = newRRAHeader();
    
    readRRAHeader(in,h,0);

    writeRRAHeader(out,h,"modifiedBy: clip",0);

    for (i = 0; i < h->samples; ++i)
        for (j = 0; j < h->channels; ++j)
            {
            amp = readRRAAmplitude(in,out,h->bitsPerSample,0);
            if (amp < - clip)
                fprintf(out,"%d\n",- clip * amplify);
            else if (amp > clip)
                fprintf(out,"%d\n",clip * amplify);
            else
                fprintf(out,"%d\n",amp * amplify);
            }

    fclose(in);
    fclose(out);

    return 0;
    }

/* only -oXXX  or -o XXX options */

static int
processOptions(int argc, char **argv)
{
    int argIndex;
    int argUsed;
    int separateArg;
    char *arg;

    argIndex = 1;

    while (argIndex < argc && *argv[argIndex] == '-')
        {
        separateArg = 0;
        argUsed = 0;

        if (argv[argIndex][2] == '\0')
            {
            arg = argv[argIndex+1];
            separateArg = 1;
            }
        else
            arg = argv[argIndex]+2;

        switch (argv[argIndex][1])
            {
            case 'a':
                amplify = atof(arg);
                argUsed = 1;
                break;
            case 'c':
                clip = atoi(arg);
                argUsed = 1;
                break;
            case 'h':
                printf("clip options:\n");
                printf("  -a N   set amplification level to N\n");
                printf("         default value is %d\n",amplify);
                printf("  -c N   set clip level to N\n");
                printf("         default value is %d\n",clip);
                printf("  -h     help\n");
                exit(0);
                break;
            case 'v':
                printf("%s version %s\n", PROGRAM_NAME, PROGRAM_VERSION);
                exit(0);
                break;
            default:
                Fatal("option %s not understood\n",argv[argIndex]);
            }

        if (separateArg && argUsed)
            ++argIndex;

        ++argIndex;
        }

    return argIndex;
    }
\end{verbatim}

One might use this program to create a set of fuzzy guitar notes from
a set of clean notes:

\begin{verbatim}
    slbuser@songlib:~/notes$ for s in clean*.rra
    > do
    > cat $s | clip -t1000 -a15 > out.rra
    > mv out.rra $s
    > done
\end{verbatim}

\section*{An analysis of the program}

The first RRA-specific function to be called in the above program is:

\begin{verbatim}
    RRA *newRRAHeader(void);
\end{verbatim}

This function returns an empty RRA object. We use the variable {\it h} to
point to the newly created object:

\begin{verbatim}
    h = newRRAHeader();
\end{verbatim}

The next RRA function called is:

\begin{verbatim}
    void readRRAHeader(FILE *,RRA *,void (*)(RRA *,char *,void *));
\end{verbatim}

Its job is to read the header information from the input file and
place it into the empty RRA object;

\begin{verbatim}
    readRRAHeader(in,h,0);
\end{verbatim}

The last argument of the function is the handler for attributes that
{\it readRRAHeader} doesn't recognize. A zero argument means that the default
unknown attribute handler is used. These attribute value pairs are placed
by the default handler
in a field called {\it items}.

The next function called is:

\begin{verbatim}
    void writeRRAHeader(FILE *,RRA *,char *,void (*)(FILE *,RRA *));
\end{verbatim}

This call begins the output of the transformed RRA file. In the program
above, the call:
    
\begin{verbatim}
    writeRRAHeader(out,h,"modifiedBy: clip",0);
\end{verbatim}

writes the header information in {\it h},
adds an additional {\it modifiedBy} field value,
and indicates that any non-standard attribute-value pairs
be written out by the default
unknown attribute handler, as indicated by the zero value.

Finally, we have
the read/write loop that processes the amplitude values in the input
RRA file:

\begin{verbatim}
for (i = 0; i < h->samples; ++i)
    for (j = 0; j < h->channels; ++j)
        {
        amp = readRRAAmplitude(in,out,h->bitsPerSample,outputComment);
        if (amp < - clip)
            fprintf(out,"%d\n",- clip * amplify);
        else if (amp > clip)
            fprintf(out,"%d\n",clip * amplify);
        else
            fprintf(out,"%d\n",amp * amplify);
        }
\end{verbatim}

The function:

\begin{verbatim}
    (int) readRRAAmplitude(FILE *,FILE *,int,void (*)(FILE *,FILE *));
\end{verbatim}

is used to read in RRA amplitude values. Since these values can be either
integer or real and can contain comments, the {\it readRRAAmplitude} function
is preferred over:

\begin{verbatim}
    fscanf(in,"%d",&amp);
\end{verbatim}

The first argument to the function is the input file pointer, the second
is the output file pointer (used by the last argument), the third argument
is the number of bits per sample, and the last argument is the comment
handler. The number of bits per sample is used to convert a
real amplitude (generally in the range from -1.0 to 1.0) to an
integer value. In the example program, the actual call is:

\begin{verbatim}
    amp = readRRAAmplitude(in,out,h->bitsPerSample,outputComment);
\end{verbatim}

The {\it outputComment} function is used to immediately write any comments
encountered to output. If one wished to change the clipping level
on the fly, a custom comment handler would be passed instead of
{\it outputComment}. This custom handler would read in the comment and,
if it was meant for the clipping program, parse it and modify the
clipping level. If the comment was not meant for the clipping program,
the custom handler should write the comment to output.

There are other functions for manipulating RRAs. They are:

\begin{center}
\begin{tabular}{ll}%
\T\toprule
{\it function~~~~~~~~~~~~~~~~~~~~~~~~~~~~~~~~~~~~~~~~~~~~~~~~~~~~~~~~~~~~~~~~~~~~~~~~~~~~~~~~~~~~~~~~~~}
& {\it purpose}\\
\T\midrule
\verb!RRA *newRRA(int,int,int,int)!
& create/return an empty {\sc RRA} object with a data portion -- the first argument
is the sample rate, the second is the number of bits per sample, the third is
the number of channels, and the fourth is the number of samples per channel\\
\verb!RRA *readRRA(FILE *,void (*)(RRA *,char *,void *))!
& read an {\sc RRA} file/return an {\sc RRA} object --
the first argument is the input file pointer,
the second is a comment handler (you can pass in zero to get the default
handler)\\ 
\verb!void createRRAData(RRA *)!
& adds a data portion to an {\sc RRA} header --
the argument is an {\sc RRA} header that has had the number of channels and the
number of samples per channel set; the size of the data portion will
be the number of channels times the number of samples per channel\\
\verb!void clearRRAData(RRA *)!
& set all sample values to zero --
the argument is the {\sc RRA} object to be cleared\\
\verb!voidfreeRRA(RRA *)! & free an {\sc RRA} object,
including the data portion --
the argument is the {\sc RRA} object to be cleared\\
\verb!RRA *cloneRRA(RRA *,RRA_TAG *(*)(RRA_TAG *))!
& clones an RRA object -- 
the first argument is the RRA object to be cloned,
the second is a handler which processes non-standard tags
(you can pass in zero to get the default tag handler, which
just copies the non-standard tags)\\
\T\bottomrule
\end{tabular}
\end{center}

\end{document}
