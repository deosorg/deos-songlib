\documentclass{article}  

\usepackage{hyperlatex}
\usepackage[margin=1.0in]{geometry}
\usepackage{color}
\usepackage{booktabs}
\usepackage[compact]{titlesec}
\usepackage{multirow}
\usepackage{upquote}
\T\usepackage{see}
\W\newcommand{\see}[1]
    {
    \IfFileExists{#1.tex}
	{\textcolor{blue}{#1}}
	{\textcolor{red}{#1?}}
    }
\newcommand{\seed}[2]
    {
    \IfFileExists{#1.tex}
	{\textcolor{blue}{#2}}
	{\textcolor{red}{#2?}}
    }


\T\begin{ifhtml}
\HlxEval{
(put 'HlxSee  'hyperlatex 'hyperlatex-See)
(put 'HlxSeed 'hyperlatex 'hyperlatex-Seed)

(defun hyperlatex-See ()
    (let ((name (hyperlatex-parse-required-argument)))
	(cond
	    ((file-exists-p (concat name ".tex"))
		(hyperlatex-insert-URL name name)
		)
	    (t
		(hyperlatex-insert-name name "red")
		)
	    )
	)
    )
(defun hyperlatex-Seed ()
    (let* ((name (hyperlatex-parse-required-argument))
	  (nickname (hyperlatex-parse-required-argument)))
	(cond
	    ((file-exists-p (concat name ".tex"))
		(hyperlatex-insert-URL name nickname)
		)
	    (t
		(hyperlatex-insert-name nickname "red")
		)
	    )
	)
    )
(defun hyperlatex-insert-name (name color)
    (insert hyperlatex-meta-< "font color=" color hyperlatex-meta->)
    (insert name "?")
    (insert hyperlatex-meta-< "/font" hyperlatex-meta->)
    )
(defun hyperlatex-insert-URL (name nickname)
    (insert hyperlatex-meta-< "a href=" hyperlatex-meta-dq)
    (insert name ".html" hyperlatex-meta-dq)
    (insert hyperlatex-meta->)
    (insert nickname)
    (insert hyperlatex-meta-< "/a" hyperlatex-meta->)
    )
}
\newcommand{\see}{\HlxSee}
\newcommand{\seed}{\HlxSeed}
\T\end{ifhtml}


\newcommand\songlib {{\it\bf songlib}}
\newcommand\Songlib {{\it\bf Songlib}}
\W\newcommand\sc\textsc

\htmldepth{1}
\htmladdress{\xlink{BBonds@cs.ua.edu}{mailto:BBonds@cs.ua.edu}}
\htmlcss{lusth.css}

\T\setlength\parskip{6 pt}
\T\setlength\parindent{0 in}
\htmltitle{Songlib: tabs}

\title{The \Songlib\ Tab Processor\\
\date{Revision Date: \today}}

\author{written by: Brandon Bonds}

\begin{document}

\maketitle

\W\subsubsection*{\xlink{Printable Version}{tab.pdf}}
\W\htmlrule

The \songlib\ Tab Processor, {\it processtab}, parses a {\it tab} file into
\songlib-based source code.  You can run {\it processtab} as follows:

\begin{verbatim}
    processtab [filename].tab
\end{verbatim}

This produces a C file, [{\it filename}].{\it c}, that builds against \songlib,
and a {\it makefile}.  (View more options by using the \verb!-h! switch.)
Running:

\begin{verbatim}
    make
\end{verbatim}

will compile [{\it filename}].{\it c} into a binary (simply named
[filename] with no extension).  The makefile will then run the binary
to produce [filename].rra.  By default, make will immediately play the
RRA file.

\section*{Tab Format Overview}

A {\it tab} file is text.  To start, let's define a very simple tablature:

\begin{verbatim}
    |SN|o-o-|oooo|
    |  |1234|1234|
\end{verbatim}

This is a single {\it stave} of tablature containing two measures of snare drum 
beats.  The first measure has notes at beats 1 and 3 and rests at beats 2 and 4.
The second measure has notes at all four beats.

However, this file is not quite complete.  {\it processtab} does not natively
understand what ``SN'' represents.  We must define it using an {\it attribute}.
The following is a complete tab file which contains a definition for
``SN'' using the \verb!voice! attribute:

\begin{verbatim}
    # Voices

    voice:
    {
        "name": "SN",
        "instrumentPath": "/usr/local/share/samples/drums/",
        "instrumentBaseName": "hera_",
        "notes":
        [
            {
                "name": "o",
                "offset": 57
            }
        ]
    }

    # Tab

    |SN|o-o-|oooo|
    |  |1234|1234|
\end{verbatim}

For now, it is sufficient to understand that a voice is defined by a name
(``voice''), followed by a colon (:), followed by JSON-formatted details inside
curly braces.

In addition to staves and attributes, tab files may also contain comments
which are not parsed.  A comment begins with the hash character (\#) and lasts
until the end of the line.

\section*{Tab Defaults}

By default, the following values are used:

\begin{itemize}
\item[stride] 0.05
\item[sustain] 0.99995
\item[primary emphasis] 1 (no emphasis)
\item[secondary emphasis] 1 (no emphasis)
\item[tempo] 80
\item[time signature] 4/4
\item[use random sampling] true
\end{itemize}

\section*{More on Staves}

A tab contains one or more {\it staves}.  Staves define the notes to be
played.  The example above contains a single stave defining notes for voice
``SN''.

Each stave ends with a single line called the {\it timing course}.  This line
defines the beat.  In the example, both measures have a basic four-beat timing
course.  More complex timing courses are possible.  The example could have been
written as either of the following:

\begin{verbatim}
    |SN|o---o---|o-o-o-o-|
    |  |1&2&3&4&|1&2&3&4&|
    
    |SN|o-------o-------|o---o---o---o---|
    |  |1e&a2e&a3e&a4e&a|1e&a2e&a3e&a4e&a|
\end{verbatim}

The first stave subdivides the timing into eighth notes, and the second stave
subdivides the timing into sixteenth notes.  Numbers indicate down beats,
ampersands (\verb!&!) represent upbeats, and ``e'' and ``a'' represent sixteenth
divisions.  This makes more intricate rhythms possible, such as the following:

\begin{verbatim}
    |SN|o--oo---|o-o-o-o-oo-oo-o-|
    |  |1&2&3&4&|1e&a2e&a3e&a4e&a|
\end{verbatim}

You may mix timings, even in the same measure:

\begin{verbatim}
    |SN|o--oo-oooo|
    |  |1&2&3&4e&a|
\end{verbatim}

You may insert dots (.) in the timing course to subdivide into triplets or
practically any other possible timing (such as sextuplets, quintuplets, etc.):

\begin{verbatim}
    |SN|o--o--oooo--|o--o--ooooooo--|
    |  |1..2..3..4..|1..2..3.....4..|
\end{verbatim}

Each beat is subdivided into triplets, except beat three of the second measure,
which is subdivided into sextuplets.

Multiple voice lines (voice courses) may exist for each stave.  Everything
should line up vertically.  Spaces are optional around the name, and may be
useful for lining up courses whose names are different lengths.  Notes that are
played at the same time must line up vertically.  All pipe characters (|) must
line up vertically.

For instance, the following stave must include an extra space for the ``B''
voice to make it line up with the ``SN'' voice:

\begin{verbatim}
    |SN|--o---o-|o-o-o-o-oo-oo-o-|
    |B |o--oo---|o---o---o---o---|
    |  |1&2&3&4&|1e&a2e&a3e&a4e&a|
\end{verbatim}

\subsection*{Notes and Rests}

Inside a measure, a dash character (-) indicates a rest and a space indicates
that the previous note is held out during that duration.  The following two
measures have the same rhythm, but differ in how long each note is held out:

\begin{verbatim}
    |SN|o o o o |o-o-o-o-|
    |  |1&2&3&4&|1&2&3&4&|
\end{verbatim}

In the first measure, each note holds out a quarter-note length.  In the second
measure, each note holds out an eighth-note length, followed by an eighth-note
rest.

``o'' has been used to indicate playing a note in each of these examples.
Any character other than space or dash may be used to represent a note, and is
defined by the voice attribute.  In the next example, we will use ``x'' to
represent a closed hi-hat, ``o'' to represent and open hi-hat, and ``@'' to
represent a rim click:

\begin{verbatim}
    |HH|XxxxXxxo|XxxxXxxo|
    |SN|--@---@-|--@---@-|
    |  |1&2&3&4&|1&2&3&4&|
\end{verbatim}

Capitalized characters typically represent accents.

\section*{More on Attributes}

An attribute is defined on one or more lines.  It lists the name of the
attribute (such as \verb!voice!), followed by a colon character (:), followed
by details in JSON (JavaScript Object Notation) format.  JSON is a common
hierarchical text format like XML, but with less bloat.

There are three built-in attribute types:  \verb!voice!, \verb!tempo!,
and \verb!amplitude!.  {\it processtab} also supports user-defined
attribute processors.

\subsection*{Voice}

The voice processor defines each voice, or instrument, that can play
notes in the tab.  It configures the voice according to the
following attributes:

\begin{description}
\item[name] (string):
    This is the unique name that also appears in the first column
    of a voice course, before measure and note definitions.  In the above
    examples, ``SN'' was defined for a snare drum.
\item[instrumentPath] (string): 
    This is the directory (folder) on disk which contains the samples.
    See the first argument for {\it readScale} in \songlib.
\item[instrumentBaseName] (string):
    This is the base name for each sample.  See the second argument
    for {\it readScale} in \songlib.
\item[amplitudeMultiplier] (float):
    This sets the base amplitude for all notes in this instrument.
    See {\it setAmplitude} in \songlib.  Note that the final amplitude
    for a note may be adjusted by other parameters, and may be adjusted by other
    processors such as {\it amplitude}.
\item[instrumentCount] (integer):
    When set higher than one, \songlib will simulate multiple
    instrumentalists by playing the note on top of itself several times.
    This is affected by the values of {\it amplitudeDeviation}, {\it
    pitchDeviation}, and {\it timingDeviation}.  The amplitude of each
    individual instrument in this set is divided by instrumentCount.
\item[amplitudeDeviation] (float):
    Sets the randomization of the amplitude.  Higher values help
    simulate instrument players with less skill.  The entire song will be
    amplitude-adjusted, up or down, at a randomized value between 1 and
    {\it amplitudeDeviation}.  When {\it instrumentCount} is greater than
    one, each individual instrument has a separate randomized amplitude
    throughout the song.
\item[pitchDeviation] (float):
    Sets the randomization of the pitch.  Higher values can simulate the
    natural difference in the pitch of percussion instruments, or help
    simulate tuned instrument players with less skill.  The entire song
    will be pitch-adjusted, up or down, at a randomized value between
    1 and {\it pitchDeviation}.  When {\it instrumentCount} is greater
    than one, each individual instrument has a separate randomized pitch
    throughout the song.
\item[timingDeviation] (float):
    Sets the randomization of timing.  Higher values can simulate mud
    or slop.  Each individual note will be offset by up to timingDeviation
    beats from its intended location.
\item[notes] (array):
    Contains several individual note definitions, each with the following
    attributes:
    \begin{description} \item[name] (character):
        A single character used to define notes in the {\it tab}.
        In the examples above, the ``o'' note is defined.
    \item[offset] (integer):
        A value indicating the offset used to play the note.  See the
        third parameter for {\it nplay} in \songlib.
    \item[accent] (boolean):
        If ``true'', this note will always be played at double the
        amplitude of a normal note.  (Typically, the {\it name} is capitalized
        for accented notes.)
    \item[delay] (integer):
        The number of samples to delay the note.  Typically this value is
        negative, causing the note to begin early.  This is useful for aligning
        the attack of a note.
    \end{description}
\end{description}

The following is a full example of a voice attribute with two notes defined:

\begin{verbatim}
	voice:
	{
		"name": "SN",
		"instrumentPath": "/usr/local/share/samples/drums-snare/",
		"instrumentBaseName": "pearlFF_",
		"amplitudeMultiplier": 16.0,
		"instrumentCount": 4,
		"amplitudeDeviation": 1.5,
		"pitchDeviation": 4.0,
		"timingDeviation": 0.125,
		"notes":
		[
		    {"name": "o", "offset": 57, "delay": -1234},
		    {"name": "O", "offset": 57, "delay": -1234, "accent": true}
		]
	}
\end{verbatim}

This voice defines a ``pearlFF'' snare drum from the ``drums-snare'' sample
pack.  The overall amplitude is adjusted louder by a factor of 16.  Each note is
played 4 times simultaneously (due to {\it instrumentCount}).  Each individually
played note has a randomly adjusted amplitude, pitch, and timings to simulate 4
real players.

Two notes are defined:  lowercase ``o'' plays sample 57 (note A4) with a delay
of -1234 samples (which makes it start playing a bit early), and uppercase ``O''
plays the same note with an accent.

\subsection*{Tempo}

Tempo is a simple processor that defines the tempo beginning at that location
in the tab.  Tempo may be redefined multiple times throughout a song as
required.

\begin{verbatim}
    # Tempo example
    tempo: { "": 164 }
\end{verbatim}

This will set the music tempo to 164 beats per minute.  See {\it setTempo} in
\songlib.

\subsection*{Amplitude}

Amplitude is a simple processor that defines the base amplitude for all voices
and notes beginning at that location in the tab.  This value multiplies against
the explicit values defined by voice attributes.  Amplitude may be redefined
multiple times throughout a song as required.

\begin{verbatim}
    # Amplitude example
    amplitude: { "": 0.015 }
\end{verbatim}

This will set the base amplitude for all instruments to 0.015.  The ``voice''
attribute will then multiply this value by any other amplitude variations it
sets.  See {\it setAmplitude} in \songlib.

\subsection*{Custom processors}

Custom processors are an advanced way to alter the output of {\it processtab}.
Custom processors are written in C and are used like built-in processors.

\subsubsection*{Writing custom processors}

We will create an example custom processor that calls the \songlib function {\it
setSkipSeconds}.  We will name this processor \verb!skipseconds!.

Each custom processor must define a {\it .c} file and a {\it .h} file.  These
files must be named the same as the processor name.  These files must define a
minimum of the following five {\it hooks} (replace {\it XYZ} with your processor
name):

\begin{itemize}
\item \verb!void setupProcessor_!{\it XYZ}\verb!(int, int, json_value *)!
\item \verb!void preProcess_!{\it XYZ}\verb!(int, Part *)!
\item \verb!void postProcess_!{\it XYZ}\verb!(int, Part *)!
\item \verb!void preRenderNote_!{\it XYZ}\verb!(int, Part *, int)!
\item \verb!void renderNote_!{\it XYZ}\verb!(int, Part *, int)!
\end{itemize}

{\it Hooks} are functions called by other code.  In our case, these hooks are
called by the output of {\it processtab}.

This is our {\it skipseconds.h} file:

\begin{verbatim}
	#ifndef INCLUDED_SKIPSECONDS_H
	#define INCLUDED_SKIPSECONDS_H

	/*
	 * Defines a skip seconds processor
	 */

	#include <tab/tabproc.h>

	extern void setupProcessor_skipseconds(int, int, json_value *);

	/* Pre- and post- processing */

	extern void preProcess_skipseconds(int, Part *);
	extern void postProcess_skipseconds(int, Part *);
		
	/* Note rendering */
		
	extern void preRenderNote_skipseconds(int, Part *, int);
	extern void renderNote_skipseconds(int, Part *, int);

	#endif /* INCLUDED_SKIPSECONDS_H */
\end{verbatim}

All of these functions must be implemented in the {\it .c} file, but any hook
that is not needed may contain an empty implementation.

In the case of \verb!skipseconds!, we will only implement \verb!setupProcessor!.
This is our {\it skipseconds.c} file, including the functions that have an empty
implementation:

\begin{verbatim}
	/*
	 * Defines a skip seconds processor
	 */

	#include "skipseconds.h"

	/* Processor setup */

	void
	setupProcessor_skipseconds(int parameterIndex, int sectionNumber, json_value *parameter)
		{
		if (getSkipSeconds() != 0.0)
			{
			Fatal("skipseconds may only be set once\n");
			return;
			}
		double seconds = get_json_double(parameter, "", 0);
		setSkipSeconds(seconds);
		}

	/* Pre- and post- processing */

	void
	preProcess_skipseconds(int parameterIndex, Part *part)
		{
		}

	void
	postProcess_skipseconds(int parameterIndex, Part *part)
		{
		}
	 
	/* Note rendering */
		
	void
	preRenderNote_skipseconds(int parameterIndex, Part *part, int noteNumber)
		{
		}
		
	void
	renderNote_skipseconds(int parameterIndex, Part *part, int noteNumber)
		{
		}
\end{verbatim}

\verb!setupProcessor! will extract the number of seconds, as a \verb!double!,
from the JSON-formatted attribute (using the {\it get\_json\_double} function).
It will then pass that value into the \songlib function {\it setSkipSeconds}.
And we will add a check using \songlib function {\it getSkipSeconds} to ensure
that the value is only set once.

\subsubsection*{Using custom processors}

Custom processors are used just like any built-in processor, by adding an
attribute at the appropriate location in your tab file.  The {\it skipseconds}
processor may be invoked as follows:

\begin{verbatim}
	skipseconds: { "" : 120 }
\end{verbatim}

{\it processtab} automatically recognizes when a tab uses custom processors (any
attribute name other than \verb!voice!, \verb!tempo!, and \verb!amplitude!), and
outputs a makefile that references the {\it .c} and {\it .h} you create for your
custom processor.

\subsubsection*{More custom processing details}

This section explains, in detail, the five hooks that are used in custom
attribute processing.

(It may help to visualize this process as it is being described.  To do so, run
{\it processtab} on a simple tablature, and read the {\it main()} function in
the output {\it .c} file.  You can see each of these hooks being called as
described below.)

\paragraph*{setupProcessor}

\verb!setupProcessor! is called first for each attribute.  This is called before
{\it songInit}, before defaults are set, and before the {\it .rra} file is open
for output.  {\it setupProcessor} is the only location that has access to the
processed JSON information.

Because of this, we may need to store any data for later use by the other four
functions.  The first integer passed into \verb!setupProcessor! is {\it
parameterIndex}, which should be associated with that data.  That same integer
is passed into the other hooks to reference that particular data.

The second parameter is the section number.  This allows attributes to affect
different areas of the song.  In this context, a {\it section} is equivalent to
the stave number above which the attribute is defined.  All attributes defined
before the first stave are in section 0, and after the first stave but before
the second are in section 1, and so on.

As you can see from the example {\it skipseconds} processor, it is possible to
ignore these two values for very simple processors.

The third parameter is the data contained within the attribute.  It is
pre-parsed using the {\it json-parser} library (see
https://github.com/udp/json-parser, and for more examples see the source code
to the built-in attribute processors).  The data is passed in as a {\it
json\_value} data structure.

\paragraph*{Other functions}

The rest of the functions take an integer {\it parameterIndex}, which is used to
reference data stored by {\it setupProcessor}.  They also take in a {\it Part},
which roughly represents a voice.

But remember, \verb!voice! is just another processor written the same way that
a custom processor is written.  It is a complex processor that ultimately calls
the \songlib {\it rplay} function.  You can write a custom processor and use that
instead of \verb!voice!, or you can copy and make changes to the \verb!voice!
processor to suit your needs.

So, a {\it Part} represents the notes that are to be played by a certain
instrument.  A part is defined by its name such as ``SN'' or ``B''.

Each part is processed at once.  \verb!preProcess! is called before processing
the notes, and \verb!postProcess! is called after processing all the notes.
Between, each individual note is processed in order by first calling
\verb!preRenderNote! for every attribute, and then \verb!renderNote! for every
attribute.

\verb!preRenderNote! and \verb!renderNote! have a third paramter, {\it
noteNumber}, which can be used to index into the {\it Part}.

\section*{Parser Limitations}

The {\it processtab} program currently requires 4/4 time signature for all tabs.
It requires that no spaces come before or after stave courses, or else a parsing
error will occur.  Also, you may come across some parsing bugs related to
whitespace between staves or between voices, staves, and comments.

All quarter note timings must be indicated.  This is partially due to the need
to make the tab language unambiguous.  Otherwise, the following invalid tab
could be interpreted multiple ways:

\begin{verbatim}
    # Invalid tab

    |SN|oooooo|
    |  |1&3&4&|
\end{verbatim}

Here, is the first \verb!&! the upbeat of beat one or the upbeat of beat two?

If ``e'' and/or ``a'' are indicated, \verb!&! must also be indicated.  However,
``e'' may be indicated without ``a'', and ``a'' without ``e'', of the same beat.

\end{document}
