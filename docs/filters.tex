\documentclass{article}  
\input{header}

\htmltitle{Songlib: filters}

\title{Songlib: filters\\
\date{Revision Date: \today}}

\author{Song Li Buser}


\begin{document}

\maketitle

\W\subsubsection*{\xlink{Printable Version}{filters.pdf}}
\W\htmlrule

\section*{Filtering notes}

It is possible to modify notes using the {\it\bf songlib} filtering system.
It is also possible to design your own filters. The basic filtering
strategy is to add filters to an instrument. Notes are then processed by
the filters when they are played. Filters are applied in the opposite order
they were placed upon
upon the instrument's filter stack.
As an example, suppose we wish to pass each note of a guitar throw a
low-pass filter (to remove any high frequency harmonics),
then a diminishing filter (to have the note fade away),
and
finally a reverb filter (to add richness).
We would push filters onto the filter stack in
the following order.

\begin{verbatim}
    guitar = readScale(GuitarDir,GuitarPrefix);

    pushFilter(guitar,myReverb);
    pushFilter(guitar,myDiminish);
    pushFilter(guitar,myLowPass);
\end{verbatim}

The added filters must have a prototype similar to...

\begin{verbatim}
    void filter(int *,int);
\end{verbatim}

...and are called back during filter processing. The added filters are
passed the raw audio data (as an integer array) and the length of the
array. The filter should then modify the audio data (in place) in the
appropriate way.

Since the
\seed{builtInFilters}{built-in filters}
take a varying number of arguments, one must
{\it wrap} them so to accomodate the two argument requirement of added
filters. Here is an example, with, {\it myLowPass} defined as follows:

\begin{verbatim}
    static void
    myLowPass(int *data,int length)
        {
        lowPass(data,length,4000,1.414);
        }
\end{verbatim}

The {\it lowPass} function is a one of the 
\seed{builtInFilters}{built-in filters}.
It
takes two additional arguments beyond the raw data and the lentgh. The
{\it myLowPass} procedure is a wrapper so that the {\it lowPass} function
can be added to the stack of instrument filters.

Filtering occurs when notes are played. Thus if a note is never
played, it is never filtered. In additions, notes are only filtered
if a non-empty filter stack is in existence at the time the
note is played. Finally, notes are (nominally) filtered once and
cached.  If the filter stack is changed and the note is replayed, the
cached version will be used and the note will not be refiltered.
If you wish to refilter a note, you must first clear it first via a call
similar to:

\begin{verbatim}
    setFilteredNote(instrument,octave,pitch,0);
\end{verbatim}

Here are the list of functions for manipulating filters and filtered
notes:

\begin{verbatim}
    void pushFilter(int instrument,filter_t filter);

    filter_t popFilter(int instrument);
\end{verbatim}

The {\it pushFilter} function adds the given filter to the stack of
filters assigned for the given instrument. The {\it popFilter} function
removes the most recently added filter still on the filter stack.
The type {\it filter\_t} is defined as:

\begin{verbatim}
    typedef void (*)(int *,int) filter_t
\end{verbatim}


One can retrieve filtered notes for use in the {\it r}-type note
playing functions:

\begin{verbatim}
    RRA *getFilteredNote(int instrument,int octave,int pitch);

    RRA *setFilteredNote(int instrument,int octave,int pitch, RRA *newest);

    RRA *getFilteredNumberedNote(int instrument,int numberedNote);

    RRA *setFilteredNumberedNote(int instrument,int numberedNote, RRA *newest);
\end{verbatim}

The {\it getFilteredNote} and {\it getFilteredNumberedNote}
functions retrieve the
specified filtered note while the {\it setFilteredNote} and
{\it setFilteredNumberedNote} update the filtered note.

The {\it set...} functions return the previous value of the filtered note.

One can also filter notes without playing them by using the
following functions:

\begin{verbatim}
    void filter(int instrument,int octave,int pitch);

    void nfilter(int instrument,int numberedNote);

    void hfilter(int instrument,RRA *h);

    void dfilter(int instrument,int *data,int length);
\end{verbatim}

Be aware that the {\it filter} and {\it nfilter} functions
permanantly filter (for the remaining life of the songlib program)
the specified clean notes.
The {\it hfliter} function filters the given RRA object while the
{\it dfilter} function filters the given data array.
All functions
use the filter stack currently associated with the given 
instrument. If the instrument's filter stack is empty, these
functions have no effect.

See also: 
\see{builtInFilters}
\end{document}
