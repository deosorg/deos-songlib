\documentclass{article}  
\input{header}

\htmltitle{Songlib: keeping time}

\title{Songlib: keepingTime\\
\date{Revision Date: \today}}

\author{written by: Song Li Buser}


\begin{document}

\maketitle

\W\subsubsection*{\xlink{Printable Version}{template.pdf}}
\W\htmlrule

\section*{Time signatures in \songlib}

\Songlib supports simple time signatures via the functions:

\begin{verbatim}
    void setTempo(int beatsPerMinute);
    void setTime(int beatsPerMeasure,int noteValue);
\end{verbatim}

Setting the tempo fixes how long a single beat lasts. The default
value is 132 beats per minute. Setting the beats per minute to a
lower value slows down the resulting output. Setting the beats
per minute to a higher value speeds up the output.

The {\it setTime} function is used to modify the time signature of
the song. The first argument,
{\it beatsPerMinute}, is used for counting measures (see below),
while the second argument, {\it noteValue}, in conjunction with
the beats per minute, determines the how long a particular note
will last. \Songlib defines the the
following note length variables:

\begin{verbatim}
    T     a thirty-secondth note
    S     a sixteenth note
    I     an eighth note (can't use E - it a key signature constant)
    Q     a quarter note
    H     a half note
    W     a whole note
\end{verbatim}

Valid second arguments to {\it setTime} are powers of 2. By taking
the reciprocal of
the second argument, one can determine which note length is equivalent
to a single beat. For example, if the second argument is 4,
then a quarter note lasts a single beat. If the second argument
is 8, then an eighth note lasts a single beat.

Here is an example,

\begin{verbatim}
    setTempo(80);
    setTime(4,4);
    play(H,instrument,3,C);
\end{verbatim}

Since the tempo is set to 80 beats per minute, one beat lasts
0.75 seconds. Since the time signature is 4:4, a quarter note
gets a single beat. The C note is then played for a half
note (which is twice the length of a quarter note) or
1.5 seconds.

The above code is equivalent to...

\begin{verbatim}
    play(1,instrument,3,C);
\end{verbatim}

...since a quarter note gets one beat according to the time signature.

The algorithm for setting the note length variables is:

\begin{verbatim}
    T  = noteValue / 32.0;
    S  = T * 2.0;
    I  = S * 2.0;
    Q  = I * 2.0;
    H  = Q * 2.0;
    W  = H * 2.0;

    //dotted notes
    Sd = S * 1.5;
    Id = I * 1.5;
    Qd = Q * 1.5;
    Hd = H * 1.5;
    Wd = W * 1.5;

    //triplet notes       //downbeat notes       //upbeat notes
    Tt = S / 3.0;         TD = 4 * T / 3.0;      TU = 2 * T / 3.0;
    St = I / 3.0;         SD = 4 * S / 3.0;      SU = 2 * S / 3.0;
    It = Q / 3.0;         ID = 4 * I / 3.0;      IU = 2 * I / 3.0;
    Qt = H / 3.0;         QD = 4 * Q / 3.0;      QU = 2 * Q / 3.0;
    Ht = W / 3.0;         HD = 4 * H / 3.0;      HU = 2 * H / 3.0;
    Wt = 2 * W / 3.0;     WD = 4 * W / 3.0;      WU = 2 * W / 3.0;
\end{verbatim}

Note that there are dotted versions 
(triplets,, downbeats, and upbeats too) of the common note lengths.
For example, the note length
{\tt Wd} is a dotted whole note (which is
equivalent to W + H),
while {\tt Wt} is a triplet
whole note (which is equivalent to W / 3).

One should always use the symbolic note lengths (W, H, Q, etc.)
instead of absolute numbers. Otherwise, adjusting the
tempo or time signature will have no affect on those
notes.

\section*{Counting measures}

Knowing how many measures a passage of a song
takes is extremely useful in aligning harmonic
and percussive tracks.
The first argument to {\it setTime} sets how
many beats are in a measure.
Given a time signature (the default time signature is 4:4),
one can count measures with the {\it measure} function:

\begin{verbatim}
    void measure(char *file,char *function,int lineNumber);
\end{verbatim}

The {\it measure} function (with a non-zero file argument) prints
out the number of measures
elapsed since the last call to {\it measure} (with a zero as a file
argument) as well as the total number of measures
elapsed since the beginning of the song. To simplify the use of
{\it measure}, two macros are defined. A typical usage of these macros
are:

\begin{verbatim}
    static void
    verse ()
        {
        startMeasure();
        play(Hd,harp,4,Cs)
        play(Q,harp,4,Ds)
        play(H,harp,4,Cs)
        play(H,harp,4,Ds)
        checkMeasure();
        play(W,harp,4,Cs)
        play(W,harp,4,F)
        checkMeasure();
        }
\end{verbatim}

The calls to {\it checkMeasure} print out information like:

\begin{verbatim}
    high.c: verse:   77: 2.000000 measures (34.000000 total)
    high.c: verse:   93: 2.000000 measures (36.000000 total)
\end{verbatim}

The definitions of {\it startMeasure} and {\it checkMeasure} are:

\begin{verbatim}
    #define startMeasure() measure(0,0,0)
    #define checkMeasure() measure(__FILE__,__FUNCTION__,__LINE__)
\end{verbatim}

You should use {\it startMeasure} (and {\it checkMeasure}) at the beginning
(and end) of every unit (verse,refrain,bridge).

\end{document}
